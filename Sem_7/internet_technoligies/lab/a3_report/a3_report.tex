\documentclass[titlepage]{article}
\usepackage{minted}
\usepackage[margin=0.5in]{geometry}
\usepackage{graphicx}

\title{Internet Technologies Lab Report\\Assignment 3}
\author{Md Sahil\\BCSE IV\\Roll-001710501029}
\date{}

\begin{document}
    {\maketitle}

    \section{Problem Statement}

    Implement a web application for “Travel Thru Air” using servlets
    to support the following two use cases

    \begin{itemize}
        \item A list of current special deals must appear on the home page.
            Each special deal must display the departure
            city, the arrival city, and the cost. These special deals are
            set up by the marketing department and change
            during the day, so it can’t be static. Special deals are only
            good for a limited amount of time.
        \item A user may search for flights, given a departure city,
            time and an arrival city. The results must display
            the departure city, the arrival city, the total cost,
            and how many legs the flight will have.
            State and explain why and where you have used design patterns.
    \end{itemize}

    \section{Design \& Implementation}

    \subsection{Directory structure}
    \begin{minted}{bash}
    travelthruair/
    |-- WEB-INF
    |   |-- classes
    |   |   |-- Makefile
    |   |   |-- api
    |   |   |   |-- ApiServlet.class
    |   |   |   `-- ApiServlet.java
    |   |   `-- model
    |   |       |-- Deal.class
    |   |       |-- Deal.java
    |   |       |-- Deals.class
    |   |       `-- Deals.java
    |   |-- lib
    |   |   |-- gson-2.8.6.jar
    |   |   `-- servlet-api.jar
    |   `-- web.xml
    |-- index.html
    `-- js
        `-- app.js

    6 directories, 12 files
    \end{minted}

    \subsection{Compilation}
    Compilation is done using the \emph{Makefile}.
    In order to compile the java classses, run the following command while in the \textbf{travelthruair/WEB-INF/classes} directory
    \begin{minted}{bash}
    $ make
    \end{minted}
    
    \pagebreak
    \subsection{Design}
    The application is build as a RESTfull service. The client side makes API calls to the server
    and the server responds accordingly.
    

    \noindent
    The api calls can be made with the url \textbf{<hostname>/travelthruair/api/deals}.\\
    Filtering of flights can be done using query params.
    The availabe query params are \emph{arrivalCity}, \emph{departureCity}, \emph{arrivalDate}, \emph{departureDate}.
    
    
    \noindent
    Example api call with query params:
    \begin{minted}{text}
        <hostname>/travelthruair/api/deals?arrivalCity=Kolkata&departureDate=2021-02-07&
    \end{minted}
    Returned json data:
    \begin{minted}{json}
    [
      {
        "departureCity": "Kolkata",
        "arrivalCity": "Delhi",
        "cost": 3000,
        "arrivalDate": {
          "year": 2021,
          "month": 2,
          "day": 8
        },
        "departureDate": {
          "year": 2021,
          "month": 2,
          "day": 7
        }
      }
    ]
    \end{minted}

    The following error checks have been implemented:
    \begin{itemize}
        \item Arrival and departure city can’t be the same
        \item Departure date and arrival date cannot be before the current date
        \item Arrival Date cannot be before departure date
    \end{itemize}

    The frontend consist of a simple form that makes calls to this api in order to fetch the
    available flights. The \emph{get deals} button returns all the upcomming available flights

    \subsection{MVC}
    The backend consists of 2 packages the \textbf{api} package and the \textbf{model} package
    The \textbf{api} package consists of the servlet classes and acts as the \textit{\textbf{Controller}}.
    The model package consists of two classes. The \textbf{Deal} class is a simple DTO (Data Transfer Object) class.
    Each object of the \textbf{Deal} class corresponds to a flight deal. 
    The \textbf{Deals} class acts as the DAO (Data Access Object) class and encapsulated the methods
    that returns the set of \textbf{Deal} objects accorting to the search params and also performs error checking.
    The \textbf{model} package acts as the \textit{\textbf{Model}} in the \textbf{MVC} pattern.
    Finally the HTML frontend that displays the contents fetched by the Javascript functions acts as the \textit{\textbf{View}}


\end{document}
